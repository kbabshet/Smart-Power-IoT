\documentclass[11pt, a4paper]{article}
\usepackage{amsmath,amsthm, graphicx}
\usepackage{verbatim}
\graphicspath{C:\Users\Trouble Maker\Documents\2015\Latex}
\title{ELEN4009 - Software Engineering\\Smart Home Power Management System\\Software Requirement Specification}
\author{Ari Croock (718005)\\Kanaka Babshet (678851)\\Alice Yang (597609)\\Daniel Weinberg (547937)}
\date{\today}

\begin{document}
	\maketitle
	\section{Introduction}
	
	This document details the system requirements specification for the Smart Home Power Management System. The system design document will be developed from this document.\\
	
	With the rapidly growing interest in new Internet of Things (IoT) technologies, networks consisting of these devices will become increasingly difficult to manage and control. Additionally, power consumption and monitoring will become a greater concern, especially in emerging markets such as South Africa.\\
	
	This project aims to provide a flexible software system which is able to remotely control and monitor IoT devices, as well as perform detailed power consumption diagnostics.
	
	\textbf{Choices:} 
	\begin{itemize}
		\item Agile SDLC - SCRUM 
		\item Architecture - Client Server - 3 tier
		\item Front-end user interface method
	\end{itemize}
	
	%3 tier has better security because 2 tier the client is allowed access and communication with 
	
	\section{External Interface Requirements}
	The requirements for the external interface will be illustrated in the following section. The 
	
	\section{Other Non-Functional Requirements}
	
	\subsection{Performance Requirements}
	The smart home power system is an application and system that is designed for efficiency. This means that the application as well as the connecting household components need to respond quickly and in real time. 
	\\\\
	The application will most likely undergo several updates and changes for the user's benefit. The system is required to update on user command. Updates will include further improvements to the application as well as fixes for issues that arise in operation.
	
	\subsection{Safety Requirements} 
	The application as well as the system components need to take certain safety concerns into account. 
	\\\\
	Due to the fact that the application is essentially controlling most of a home's appliances and electricity usage, it is important that it monitors everything it is controlling to prevent problems that may arise. The application needs to ensure that any device connected in the system does not reach a dangerous level of usage. In this case, the application should either switch off the device in question or notify the user that something is not functioning correctly and needs to be addressed. 
	
	\subsection{Security Requirements}
	Due to the fact that the system controls a user's home, it therefore requires security considerations to be taken into account.
	\\\\
	The system needs to ensure that only the user has access to the application to prevent external parties gaining control of the connected components within a house. This can be done with a signup, login and authentication process. 
	The components within the house that the application connects with, need to be protected from external parties and therefore they must be explicitly authenticated. 
	
	\subsection{Software Quality Attributes}
	The application is web based and needs to be user friendly. The functioning of the application needs to be simple so that no additional documentation or prior knowledge or experience is required. 
	
	
\end{document}